%!TEX TS-program = xelatex
\documentclass[12pt, a4paper, oneside]{article}

% Можно вставить разную преамбулу
\input{preamble}

% для нормального распределения
\newcommand{\expp}[1]{ \exp \left( #1 \right)} 
% для прорисовки нормального распределения
\newcommand\gauss[2]{1/(#2*sqrt(2*pi))*exp(-((x-#1)^2)/(2*#2^2))} 


\title{\begin{center} \includegraphics[width=0.99\textwidth]{logo.png} \end{center}  Каждой Маше по три медведя! \footnote{Эта pdf-ка, по факту, представляет из себя кусочек недописанной виньетки по Байесовским методам: \newline  \url{https://github.com/FUlyankin/book_about_bayes}}}
% \author{Ульянкин Филя, Романенко Саша}

\begin{document}
	
	\maketitle
	
\epigraph{Идет медведь по лесу, видит, машина горит. Сел в нее и сгорел.}{Анекдот категории F}
	
\begin{problem}{(Маша и медведи)}
Маша прячется от Медведей в точке $m$ на числовой прямой. Есть несколько Медведей, каждый из которых обнюхивает всю числовую прямую в поисках Маши. Медведю номер $i$ кажется, что Машей сильней всего пахнет в точке $y_i$. Естественно, Медведи могут ошибаться, например, у них может быть заложен нос, поэтому \indef{модель Медведя выглядит как:}

\[ y_i \mid m \sim \mN(m, 2^2).\]

При фиксированном $m$ величины $y_i$  независимы. Известно, что $y_1 = 0.5$, $y_2 = −1$.  Априорно известно, что место, где спряталась Маша имеет нормальное распределение, $m \sim \mN(1, 4^2)$. Нам нужно:

\begin{enumerate}
	\item Найти апостериорную плотность распределения параметра $m$.
	\item Найти апостериорные моду, медиану и математическое ожидание.
	\item Найти $\PP(m > 1 \mid y_1,y_2)$.
	\item Найти $f(y_3 \mid y_1,y_2)$ и $\E(y_3 \mid y_1,y_2)$.
\end{enumerate}
\end{problem}

\begin{sol}
Посмотрим немного подробнее на наше априорное мнение о том, где сидит Маша, $m \sim \mN(1, 4^2)$. Значение $1$ в данном случае --- наше лучшее предположение о том, где она может находиться, а $4^2$, в свою очередь, это наша степень доверия к этому предположению. Чем меньшее значение дисперсии мы берём в нашем априорном мнении, тем больше наше доверие к нему.

\indef{Делай раз! Апостериорная плотность Маши:}

\begin{multline*}
f(m \mid y_1, y_2) \propto f(y_1,y_2 \mid m) \cdot f(m) = \frac{1}{2\sqrt{2\pi}}\expp{-\frac{(0.5 - m)^2}{2 \cdot 4}} \cdot \\ \cdot \frac{1}{2\sqrt{2\pi}}\expp{-\frac{(-1 - m)^2}{2 \cdot 4}} \cdot \frac{1}{4\sqrt{2\pi}}\expp{-\frac{(m-1)^2}{2 \cdot 16}}
\end{multline*}

Воспользуемся магической силой уже привычного нам значка $\propto$ и для простоты расчётов пренебрежём кучей констант

\[ f(m \mid y_1, y_2) \propto \expp{-\frac{(0.5 - m)^2}{2 \cdot 4}} \cdot \expp{-\frac{(-1 - m)^2}{2 \cdot 4}} \cdot \expp{-\frac{(m-1)^2}{2 \cdot 16}}.\]

Сольём всё,что находится под знаком экспоненты в единое целое и упростим

\begin{multline*}
\frac{(0.5 - m)^2}{2 \cdot 4} + \frac{(-1 - m)^2}{2 \cdot 4} + \frac{(m-1)^2}{2 \cdot 16}  = \\ = \frac{ 4(m - 0.5)^2 + 4(m+1)^2 + (m-1)^2}{32} = \frac{9m^2 + 2m + 6}{32}
\end{multline*}

Используем двойную магию. С одной стороны пренебрегаем константой, с другой создаём новую для того, чтобы выделить полный квадрат. Не забываем перекинуть в знаменатель лишнюю девятку

\begin{multline*}
\expp{-\frac{9m^2 + 2m + 6}{32}} \propto \expp{-\frac{9m^2 + 2m}{32}}  = \\ = \expp{-\frac{m^2+\frac{2}{9} m}{\frac{32}{9}}}  = \expp{-\frac{m^2+ 2 \cdot \frac{1}{9} m + \frac{1}{81} - \frac{1}{81}}{\frac{32}{9}}} \propto  \\ \propto  \expp{-\frac{m^2+ 2 \frac{1}{9} m + \frac{1}{81}}{\frac{32}{9}}} = \expp{-\frac{(m + \frac{1}{9})^2}{2 \cdot (\fr{4}{3})^2}}
\end{multline*}

Видим, что параметр $m$ имеет нормальное апостериорное распределение

\[m \mid y_1, y_2 \sim \mN(-\fr{1}{9},(\fr{4}{3})^2).\]

При желании можно восстановить константу. Обратите внимания, что после того как Медведи попытались вынюхать, где находится Маша, самое вероятное её положение изменилось, а дисперсия её положения уменьшилась.

\begin{figure}[h!]
	\begin{center}
		\begin{tikzpicture}[scale = 1.2,
		every pin edge/.style={<-},
		every pin/.style={fill=yellow!50,rectangle,rounded corners=3pt,font=\small}]
		\begin{axis}[every axis plot post/.append style={
			mark=none,domain=-10:10,samples=100,smooth},
		clip=false,
		axis y line=none,
		axis x line*=bottom,
		ymin=0,
		xtick=\empty,
		]
		\addplot [blue] {\gauss{1}{2}};
		\addplot [red] {\gauss{-1/9}{4/3}};
		
		\addplot [magenta,dashed] {\gauss{0.5}{2}};
		\addplot [magenta,dashed] {\gauss{-1}{2}};
		
		\node[pin=70:{апостериорная плотность}] at (axis cs:-0.2,0.3) {};
		\node[pin=70:{априорная плотность}] at (axis cs:1.5,0.19) {};
		\node[pin=70:{вынюханная информация}] at (axis cs:1.9,0.1) {};
		\end{axis}
		\end{tikzpicture}
	\end{center}
	\caption{Информация  о Маше}
\end{figure}

Новая информация сместила априорную плотность влево и вытянула её вверх, в силу того, что Медведи вынюхали похожие вещи.

\indef{Делай два!} Мода и медиана для нормального распределения совпадают с математическим ожиданием. Мы можем использовать эти величины в качестве точечных оценок.

\indef{Делай три!} Обратите внимание, что до запуска Медведей, $\PP(m > 1) = 0.5$. После запуска, эта вероятность уменьшится,так как распределение очень сильно съедет влево.

\begin{multline*}
\PP(m > 1 \mid y_1,y_2) = 1 - \PP( m \le 1 \mid y_1,y_2) = \\ = 1 - \PP\left( \frac{m + \fr{1}{9}}{\fr{4}{3}} \le \frac{1 + \fr{1}{9}}{\fr{4}{3}} \mid  y_1, y_2 \right) = 1 - \Phi\left(\frac{10}{12}\right) \approx 0.2.
\end{multline*}

Значение функции $\Phi(z)$ можно получить, воспользовавшись таблицами для стандартной нормально распределённой случайной величины. Либо её можно найти с помощью компьютера. 


\indef{Делай четыре!} Найдём $f(y_3 \mid y_1,y_2)$ и $\E(y_3 \mid y_1,y_2)$. Будем делать это под слоганом: <<Каждой Маше по три Медведя!>>:

\begin{multline*}
f(y_3 \mid y_1,y_2)  = \int_{-\infty}^{+\infty} f(y_3, m \mid y_1,y_2)\dx{m}  = \int_{-\infty}^{+\infty}  f(y_3 \mid y_1,y_2,m) \cdot f(m \mid y_1,y_2) \dx{m}.
\end{multline*}

Под знаком интеграла мы получаем произведение модели и апостериорного распределения. Чтобы найти плотности распределение $y_3$, мы должны провести свёртку по двум нормальным распределениям

\begin{multline*}
f(y_3 \mid y_1,y_2)  = \int_{-\infty}^{+\infty} \mN(m,4)\cdot \mN \left (-\frac{1}{9},\frac{16}{9} \right) \dx{m} = \\ = \int_{-\infty}^{+\infty} 
\frac{1}{2\sqrt{2\pi \cdot 4^2}} \expp{-\frac{(y - m)^2}{2 \cdot 4^2}} \cdot 
\frac{1}{2\sqrt{2\pi \cdot (^{16}/_9)^2}} \expp{-\frac{(m + ^1/_9)^2}{2 \cdot (^{16}/_9)^2}} \dx{m} \propto \\ \propto   \int_{-\infty}^{+\infty}  \expp{-\frac{(y - m)^2}{2 \cdot 4^2} -\frac{(m + ^1/_9)^2}{2 \cdot (^{16}/_9)^2}} \dx{m}  
\end{multline*}

Если аккуратно взять этот интеграл и восстановить константу, можно получить, что $y_3 \mid y_1,y_2 \sim \mN \left(-\frac{1}{9},\frac{52}{9} \right).$ Если нам необходим точечный прогноз и в качестве функции потерь выбрано $MSE$, мы можем выбрать число $-\frac{1}{9}.$

Теперь, когда нюхательные способности третьего Машиного Медведя предсказаны, вы можете попробовать проделать всё то же самое самое, предположив, что вам вообще ничего неизвестно и $m \sim \mU(-\infty; +\infty)$. В таком случае в качестве плотности нужно будет взять $f(m) = 1$. Стоит отметить, что результат у вас, при этом, получится похожим на случай нормального априорного распределения с большой дисперсией. Эти два распределения обладают довольно большой энтропией. Из-за этого получаются схожие результаты. Также попытайтесь провернуть процедуру байесовского вывода для нормального распределения в общем случае. Формулы выйдут довольно громоздкими. Если запутаетесь, \href{https://github.com/FUlyankin/book_about_bayes}{загляните в решебник.}
\end{sol}

\end{document}