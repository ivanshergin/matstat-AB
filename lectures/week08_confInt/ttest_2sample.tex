%!TEX TS-program = xelatex
\documentclass[12pt, a4paper, oneside]{article}

% Можно вставить разную преамбулу
\input{../preamble}

\title{\begin{center} \includegraphics[width=0.99\textwidth]{logo.png} \end{center}Равенство средних, неизвестные равные дисперсии}
\date{ } %\today}
% \author{Ульянкин Филя, Романенко Саша}

\begin{document} % Конец преамбулы, начало файла

\maketitle

\epigraph{Фишер всё равно бы сумел открыть всё это сам.}{Уильям Госсет (Стьюдент) о своих открытиях в статистике}


\section*{Равенство средних, неизвестные равные дисперсии}

У нас есть две независимые выборки из нормально распределённых генеральных совокупностей: 

\begin{equation*} 
\begin{aligned} 
& X_1, \ldots, X_{n_x} \iid \mN(\mu_x, \sigma^2), \\  
& Y_1, \ldots, Y_{n_y} \iid \mN(\mu_y, \sigma^2). 
\end{aligned} 
\end{equation*} 

Дисперсии неизвестны, но одинаковы. Нужно построить доверительный интервал для разности средних на уровне значимости $\alpha$. Если бы дисперсия была известна, мы бы использовали для этого статистику 

$$
z = \frac{\bar{x} - \bar{y} - (\mu_x - \mu_y)}{\sqrt{\frac{\sigma^2}{n_x} + \frac{\sigma^2}{n_y}}} \sim \mN(0,1).
$$

Однако дисперсия неизвестна и её нужно оценить. В связи с этим возникает вопрос, каким будет распределение у новой статистики, где $\sigma^2$ заменяется на оценку, посчитанную по нашим двум выборкам. 

Если мы по каждой выборке посчитаем оценки дисперсий $s^2_x$ и $s^2_y$, то по теореме Фишера 

\begin{align*} 
\frac{(n_x - 1) \cdot s_x^2}{\sigma^2} \sim \chi^2_{n_x - 1} & & \frac{(n_y - 1) \cdot s_y^2}{\sigma^2} \sim \chi^2_{n_y - 1}. \\
\end{align*}

Распределение хи-квадрат устойчиво относительно суммирования, то есть если  $Y_1, Y_2$ независимы, и $Y_1 \sim \chi^2_{k_1}$, а $Y_2 \sim \chi^2_{k_2}$, то

$$ 
Y_1 + Y_2 \sim \chi^2_{k_1 + k_2}.
$$

Получается, что по свойствам хи-квадрат

$$
W = \frac{(n_x - 1) \cdot s_x^2}{\sigma^2}  + \frac{(n_y - 1) \cdot s_y^2}{\sigma^2} \sim \chi^2_{n_x + n_y - 2}. 
$$

Попробуем воспользоваться тем же самым приёмом, с помощью которого мы строили $t$-статистику для проверки гипотезы о среднем. Выписанная для разницы средних $z$- статистика нормально распределена, но дисперсия в знаменателе неизвестна. Попробуем поделить её на величину $W$ таким образом, чтобы неизвестная дисперсия заменилась на её оценку, а в знаменателе оказалось хи-квадрат распределение. Тогда итоговая статистика будет иметь распределение Стьюдента

\begin{equation*} 
\frac{\mN(0,1)}{ \sqrt{\frac{\chi^2_{n_x + n_y - 2}}{ (n_x + n_y - 2)}}} = t(n_x + n_y - 2).
\end{equation*} 

Получается, что

\begin{equation*} 
t = \frac{\bar{x} - \bar{y} - (\mu_x - \mu_y)}{\sqrt{\frac{\sigma^2}{n_x} + \frac{\sigma^2}{n_y}}} : \sqrt{ \frac{ [(n_x - 1) \cdot s_x^2 + (n_y - 1) \cdot s_y^2] / \sigma^2} {(n_x + n_y - 2)}} \sim t(n_x + n_y - 2).
\end{equation*} 

Общая выборочная дисперсия будет считаться по формуле 

$$
s^2 = \frac{1}{n_x + n_y - 2} \cdot \left[ (n_x - 1) \cdot s_x^2 + (n_y - 1) \cdot s_y^2 \right]. 
$$

Две степени свободы расходуются на оценку двух средних. Упростим выражение 

\begin{equation*} 
t = \frac{\bar{x} - \bar{y} - (\mu_x - \mu_y)}{\sqrt{\frac{\sigma^2 \cdot (n_y + n_x)}{n_x \cdot n_y}}} : \sqrt{ \frac{s^2} {\sigma^2}} =  \frac{\bar{x} - \bar{y} - (\mu_x - \mu_y)}{\sqrt{\frac{s^2 \cdot (n_x + n_y)}{n_x \cdot  n_y}}} \sim t(n_x + n_y - 2).
\end{equation*} 

Получилось! Когда мы поделили $z$- статистику на функцию от $W$, неизвестная дисперсия заменилась на её оценку, а итоговая случайная величина имеет известное нам распределение Стьюдента. Эту случайную величину мы можем использовать для строительства доверительных интервалов и проверки гипотез. 


%\section{Равенство средних, неизвестные неравные дисперсии}

%\textbf{t-критерий Уэлча (Welch's t-test, or unequal variances t-test)} — тест, основанный на распределении Стьюдента и предназначенный для проверки статистической гипотезы о равенстве математических ожиданий случайных величин, имеющих необязательно равные известные дисперсии. Является модификацией t-критерия Стьюдента и назван в честь британского статистика Бернарда Льюиса Уэлча.


\end{document}
