%!TEX TS-program = xelatex
\documentclass[12pt, a4paper, oneside]{article}

% Можно вставить разную преамбулу
\input{preamble}

\title{\begin{center} \includegraphics[width=0.99\textwidth]{logo.png} \end{center} Дельта-метод\footnote{Эта pdf-ка, по факту, представляет из себя немного дополненный конспект Бориса Демешева:  \url{https://github.com/bdemeshev/pr201/tree/master/delta_method}}}
\date{ } %\today}
% \author{Ульянкин Филя, Романенко Саша}

\begin{document}

\maketitle

\epigraph{<<Забавная странная цитата в тему в стиле Николенко>>}{Автор цитаты}

Нормальное распределение возникает, если суммируется большое количество независимых одинаково распределенных случайных величин. Однако оно возникает и в других ситуациях! Дельта-метод основан на том факте, что даже нелинейная функция от нормально распределенной случайной величины  иногда имеет распределение близкое к нормальному.

\section*{Откуда берётся дельта-метод}

Если функция $g(t)$ дифференцируема, то в окрестности точки $\mu$ функция $g(t)$ похожа на прямую, то есть 

$$
g(t) \approx g(\mu) + g'(\mu) \cdot (t - \mu).
$$

Об этом нам говорит математический анализ, в частности, разложение в ряд Тэйлора. 

Линейное преобразование нормально распределенной случайной величины оставляет её нормально распределенной, если угловой коэффициент отличен от нуля, т.е. 

$$
g'(\mu) \neq 0.
$$ 

Если $X \sim \mN(\mu, \sigma^2)$ и  дисперсия $X$ мала, то $X$ практически всегда попадает в небольшую окрестность $\mu$, а в ней $f$ похожа на линейную функцию и 

$$
g(X) \approx N(\mu, \sigma^2 (g'(\mu))^2.
$$ 

\indef{Получаем практическую версию дельта-метода.} Если: 

\begin{itemize}
	\item  $g(t)$ --- дифференциируема;
	\item  $g'(\mu) \neq 0$;
	\item $X \sim \mN(\mu,\sigma^2)$;
	\item дисперсия $\sigma^2$ мала;
\end{itemize} 

тогда 

$$
g(X) \sim \mN(g(\mu),\sigma^2 (g'(\mu))^2).
$$


ЗБЧ позволяет использовать средние в качестве оценок для различных параметров. ЦПТ подсказывает как среднее будет распределено. Однако на практике часто встречаются ситуации, когда оценка параметра --- это функция от среднего.  \indef{Дельта-метод ---} позволяет в такой ситуации понять как будет распределена оценка. Полученное распределение можно использовать для строительства доверительного интервала. 


\section*{Дельта-метод на практике}

\begin{problem}{(Равномерное)}
Пусть случайные величины $X_1, \ldots, X_{100} \iid U[2;8]$. Как будут распределены $\bar{x}$ и $\frac{1}{\bar{x}}$? 
\end{problem} 

\begin{sol}
С $\bar{x}$ всё будет просто. Воспользуемся ЦПТ, по ней 

$$
\bar{x}\sim \mN \left( \E(X_i), \frac{\Var(X_i)}{n} \right).
$$

Для равномерного распределения $\E(X_i) = \frac{2 + 8}{2} = 5$, $\Var(X_i) = \frac{(8-2)^2}{12} = 3$.

Получается, что среднее посчитанное по сотне наблюдений будет иметь распределение 

$$
\bar{x}_{100} \sim \mN \left( 5, \frac{3}{100} \right).
$$

Для поиска распределения $\frac{1}{\bar{x}}$ воспользуемся дельта-методом: 

$$
g(t) = \frac{1}{t} \qquad g'(t) = -\frac{1}{t^2} \qquad g(\mu) = \frac{1}{5} \qquad g'(\mu) = - \frac{1}{25}.
$$

Остаётся только подставить  найденные значения в формулу и получить, что 

$$
\frac{1}{\bar{x}_{100}} \sim \mN \left( \frac{1}{5}, \frac{3}{100} \cdot \left(-\frac{1}{25} \right)^2\right).
$$
\end{sol}

\begin{problem}{(Пуассона)}
Пусть $X_1, \ldots, X_n \iid \Pois(\lambda)$.   С помощью дельта-метода найдите как распределена оценка вероятности $\PP(X_i = 0)$.
\end{problem} 

\begin{sol}
В качестве оценки для $\lambda$ будем использовать оценку метода моментов, $\bar{x}$.  Среднее по ЦПТ имеет асимптотически нормальное распределение

$$
\bar{x}\sim \mN \left(\lambda, \frac{\lambda}{n} \right).
$$

Вероятность того, что $X_i = k$ считается по формуле 

$$
\PP(X_i = k) = \frac{\lambda^k}{k!} \cdot e^{-\lambda},
$$ 

в частности 

$$
\PP(X_i = 0) = e^{-\lambda}.
$$

Для оценки последней, $e^{-\bar{x}}$ нам нужно найти распределение. Воспользуемся  дельта-методом:

$$
g(t) = e^{-t} \qquad g'(t) = -e^{-t}
$$

Подставим значения в формулу и получим, что 

$$
e^{-\bar{x}} \sim \mN \left( e^{-\lambda},  \frac{\lambda}{n} \cdot e^{-2 \cdot \lambda}  \right).
$$

В дисперсию можем подставить вместо $\lambda$ её оценку

$$
e^{-\bar{x}} \sim \mN \left( e^{-\lambda},  \frac{\bar{x}}{n} \cdot e^{-2 \cdot \bar{x}}  \right).
$$

Такое распределение мы сможем использовать для строительства доверительных интервалов и дальнейшего анализа.
\end{sol}


\section*{Дельта-метод в теории}

Естественно, строгая формулировка идеи <<дисперсия $\sigma^2$ мала>> использует понятие предела и последовательностей случайных величин.

Если:  $g(t)$ --- дифференцируема, $g'(\mu)\neq 0$, и последовательность случайных величин $X_1, X_2, \ldots, X_n, \ldots $ удовлетворяет условию:

\[
\sqrt{n} (X_n - \mu) \overset{d}{\to}  \mN(0,\sigma^2),
\]

тогда последовательность $g(X_n)$ удовлетворяет условию:

\[
\sqrt{n} (g(X_n) - g(\mu)) \overset{d}{\to} \mN(0,\sigma^2 (g'(\mu))^2 )
\]


\end{document}
